\documentclass[a4paper, 11pt]{report}

\addtolength{\hoffset}{-1cm}
\addtolength{\textwidth}{2cm}

\usepackage[utf8]{inputenc}
\usepackage[frenchb]{babel}
\usepackage[T1]{fontenc}

\usepackage{tikz}

\usepackage{csquotes}

\usepackage{listings}
\usepackage{color}
\definecolor{lightgray}{rgb}{.9,.9,.9}
\definecolor{darkgray}{rgb}{.4,.4,.4}
\definecolor{purple}{rgb}{0.65, 0.12, 0.82}

\lstnewenvironment{OCaml}
                  {\lstset{
                      language=[Objective]Caml,
                      breaklines=true,
                      commentstyle=\color{purple},
                      stringstyle=\color{red},
                      identifierstyle=\ttfamily,
                      keywordstyle=\color{blue},
                      basicstyle=\footnotesize,
                      xleftmargin=0.15\textwidth
                    }
                  }
                  {}
\lstnewenvironment{C}
                  {\lstset{
                      language=C,
                      breaklines=true,
                      commentstyle=\color{purple},
                      stringstyle=\color{red},
                      identifierstyle=\ttfamily,
                      keywordstyle=\color{blue},
                      basicstyle=\footnotesize
                    }
                  }
                  {}

\begin{document}

\chapter{Projet de CA : \emph{ml2mips}}

Béatrice CARRE

Marwan GHANEM

Stéphane FERREIRA

\section*{Introduction}
Le projet consiste à utiliser un compilateur de mini-ML vers Java, pour créer un compilateur de mini-ML vers MIPS.

\section{Présentation}
Le travail à réaliser est de modifier le code fourni du compilateur
ml2java, pour en faire un compilateur ml2mips.
Les étapes d'analyse lexicale et syntaxique du langage mini-ML nous étaient
donc founies. Restait à faire le compilateur du langage intermédiaire
vers MIPS. 
Pour cela, nous avons dû commencer par ces différentes étapes :
\begin{itemize}
\item parcourir le code fourni
\item comprendre et prendre en main son architecture
\item en filtrer le plus utile et ce qui sera à modifier
\item approfondir nos connaissance en MIPS
\item pour chaque expression/instruction, faire un schéma de compilation
\end{itemize}
De ces étapes, nous avons pu en tirer assez d'informations pour
commencer l'implantation.

\section{brève présentation du mini-ml}
Pour les tests, il était important de comprendre les différences avec
le OCaml. Pour notre bien, ce langage était très limité.
Les mot-clés (tokens) utilisés pour ce projets sont :
\begin{OCaml}
else, function, if, in, let, rec, ref, then
\end{OCaml}
ainsi que tous les symboles habituels :
\begin{OCaml}
"=","(",",","->","::", etc.
\end{OCaml}
Voici quelques exemples de syntaxe du mini-ml :
\begin{OCaml}
let rec sub = function x -> sub;;
\end{OCaml}
est autorisé, mais pas 
\begin{OCaml}
let f = sub;;
\end{OCaml}
De plus, chaque instruction doit obligatoirement finir par ";;", comme
lors des évaluations dans le top-level.
 
\section{Le langage intermédiaire}

Mais pour l'implémentation de notre générateur de code, il a fallut
surtout bien prendre en main le langage intermédiaire, décrit dans
langinter.ml dont voici un extrait :
\begin{OCaml}
type li_type = 
  ALPHA
| CONSTTYPE of li_const_type
| PAIRTYPE
| LISTTYPE
| FUNTYPE
| REFTYPE

type li_const = 
  INT of int
| FLOAT of float
| BOOL of bool
| STRING of string
| EMPTYLIST
| UNIT

type li_instr = 
  CONST of li_const
| VAR   of string * li_type
| IF  of li_instr * li_instr * li_instr
| PRIM of (string * li_type) * li_instr list
| APPLY of li_instr * li_instr
| RETURN of li_instr
| AFFECT of string * li_instr
| BLOCK of (string  * li_type * li_instr) list  * li_instr
| FUNCTION of string * li_type * int * (string list * li_type) * li_instr 
\end{OCaml}

\section{Schémas de compilation}
Après réflexion en groupe, nous avons décidé de schémas de
compilation, indiquant comment traiter chaque instruction du langage
intermédiaire.

Nous considérons dans ces schémas de compilation le contexte suivant :
\begin{itemize}
\item fr (:bool) la pésence d'un type retour
\item sd (:string) le nom de la dernière variable mise dans la pile ???
\item nb (:int) la profondeur dans un bloc (+2 à chaque ? ) ???
\item ainsi que des variables globales du programme :
  \begin{itemize}
   \item count (:int ref) un compteur de variables 
   \item tabvar (:int list ref) la liste des variables globales
   \item dollar2(3) (:bool ref) vrai si le registre \$2(3) est est le
     dernier utilisé pour les calculs intermédiaires.
\end{itemize}
\end{itemize}
La génération de code se fait en 3 passes sur l'AST explicitées ci-dessous.

\subsection{la fonction prod\_GV}
Pour commencer, la production de variables globales, dans
\emph{prod\_GV ast\_li} :


 $[\![ VAR(v,t) ]\!]
_{(fr=false,sd="",nb=0),count,tabvar,dollar2,dollar3 | v\notin tabvar, t\neq FUNTYPE} $
$\longrightarrow$
\begin{OCaml}
tabVar := v::!tabVar;  
out_start ("\n"^v^ ":") 0 ;  
out_start (string_of_type t) 3; 
\end{OCaml}


 $[\![ CONST(INT i) ]\!]
_{(fr=false,sd="v",nb=0),count,tabvar,dollar2,dollar3} $
$\longrightarrow$
\begin{OCaml}
out ((string_of_int i));
count := !count + 1
\end{OCaml}


 $[\![ AFFECT(var, instr) ]\!]
_{(fr,sd=,nb),count,tabvar,dollar2,dollar3} $
$\longrightarrow$
\begin{center}
 $[\![ i ]\!]
_{(fr=false,sd=var,nb),count,tabvar,dollar2,dollar3} $
\end{center}

 $[\![ PRIM((s,lt),instr\_list) ]\!]
_{(fr=false,sd="v",nb),count,tabvar,dollar2,dollar3} $
$\longrightarrow$
\begin{OCaml}
let res = List.
out ((string_of_int i));
count := !count + 1
\end{OCaml}

 $[\![ BLOCK(list, instr) ]\!]
_{(fr,sd=,nb),count,tabvar,dollar2,dollar3} $
$\longrightarrow$
\begin{center}
List.iter (fun (s,li\_t,i) -> $[\![ i ]\!]
_{(fr,sd=,nb),count,tabvar,dollar2,dollar3} $) list;

$[\![ instr ]\!]
_{(fr,sd=,nb),count,tabvar,dollar2,dollar3} $
\end{center}

\subsection{la fonction prod\_main}
La production du main, dans \emph{prod\_main ast\_li}
Pour cette partie, nous ajoutons les variables globales suivantes dans
l'environnement :
\begin{itemize}
\item etiqf (:string), qui est le nom de la fonction appelée
\item tabvarf (:(string*li\_type) list)  , stock dans un couple !! TODO !!
\item iscreated(:bool), vrai si 
\end{itemize}
$[\![ CONST(c) ]\!]
_{(fr,sd=,nb),tabvarF} $
$\longrightarrow$
\begin{OCaml}
  if nb >1 then
  tabVarF := (c,"")::!tabVarF;
\end{OCaml}
     
$[\![ VAR(v,FUNTYPE) ]\!]
_{(fr,sd=,nb),etiqf,tabvarF} $
$\longrightarrow$
\begin{OCaml}
  etiqf := (String.sub v ((String.index v '.')+1) (((String.length v) - (String.index v '.'))-1) );
  isCreat := true;
  tabVarF := [];
\end{OCaml}

$[\![ VAR(v,t) ]\!]
_{(fr,sd,nb),etiqf,tabvarF} $
$\longrightarrow$
\begin{OCaml}
  if nb > 1 then
  tabVarF := (UNIT,v) ::!tabVarF;
\end{OCaml}

$[\![ AFFECT(v,i) ]\!]
_{(fr,sd,nb),etiqf,tabvarF} $
$\longrightarrow$
\begin{OCaml}
 prod\_call\_func2 (false,"pp",nb+1) i;
\end{OCaml}

$[\![ BLOCK(l,i) ]\!]
_{(fr,sd,nb),etiqf,tabvarF} $
$\longrightarrow$
\begin{OCaml}
  List.iter (fun (v,t,i) -> prod\_call\_func2 (false,sd,nb+1) i) l;
  prod\_call\_func2 (fr,sd,nb+1) i;
\end{OCaml}

$[\![ BLOCK(l,i) ]\!]
_{(fr,sd,nb),etiqf,tabvarF,isCreat} $
$\longrightarrow$
\begin{OCaml}
  if !isCreat then(
  tabVarF := List.rev !tabVarF;
  prod\_fun\_called !etiqf !tabVarF;
  tabVarF := [];
  isCreat := false
  )
\end{OCaml} 


\subsection{la fonction prod\_func}
La production des ??? de fonction, dans \emph{prod\_func ast\_li}


%out "\underbrace{   }_{nb espaces}"
\section{Difficultés et choix d'implémentation}
Lors des tests effectués, nous avons découvert que l'on peut
\begin{itemize}
\item différenciation appel fun et appel de func dans une autre CAR
  derniere instruction 
\item problème de synchronasation entre var pile et var registre.
\item 
\end{itemize}

\section{Conclusion}

\section*{Références}
\begin{itemize}
\item 
\item 
\item 
\end{itemize}

\end{document}
